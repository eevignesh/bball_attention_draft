\section{Related Work}

\JONATHAN{ I'll take a look for more papers this week.  My feeling is that
  we're missing literature about sports video analysis.  Kevin’s former student
  (Wei Lwun's) paper, for example has not been cited yet, and is a pretty
  relevant.  We might even want to see whether there are good papers cited by
  that paper.}

\noindent \textbf{Action recognition in videos}
Traditionally, well engineered features have proved quite effective for video
classification and retrieval tasks
\cite{Dalal_ECCV06,Jain_CVPR13,Jiang_ECCV12,Laptev_CVPR08,
Niebels_ECCV10,Oh_MVA14,Oneata_ICCV13,Peng_ECCV14,Sadanand_CVPR12,Wang_BMVC09,Wang_CVPR11}.
The improved dense trajectory (IDT) features \cite{Wang_CVPR11} achieve
competetive results on standard video datasets.  In the last few years,
end-to-end trained deep netowork models
\cite{Ji_PAMI13,Karpathy_CVPR14,Simonyan_2014,Tran_arxiv14} were shown to be comparable and
at times better than these features for various video tasks.  Other works like
\cite{Xu_2015,Zha_2015,Wang_arxiv15} explore methods for pooling such
features for better performance.

Recent works using Recurrent Neural Networks (RNN) have achieved
state-of-the-art results for both event recognition and caption-generation
tasks \cite{Donahue_arxiv14,Ng_arxiv15,Srivastava_2015,Yao_arxiv15}.
We follow this line of work with addition of attention mechanism
to attend to the event particiapants.

\noindent \textbf{Attention models}
Itti et al. \cite{Itti_PAMI98} explored the idea of saliency based attention in
images, with other works like \cite{Shapovalova_NIPS13} using eye-gaze data as
a means for learning attention.
Other approaches such as Gkioxari et al. \cite{Gkioxari_arxiv14} and Raptis et al. \cite{Raptis_CVPR12}
automatically localize a spatio-temporal tube in the video corresponding to the action.

Recently, \cite{Bahdnau_arxiv14} showed that
attention based RNN models can effectively align input words in a sentence to
output words for machine translation.  Following this work, attention was used
for aligning regions in an image to output words for image-cpationing
\cite{Xu_arxiv15} and frames in a video with output words for video-captioning
\cite{Yao_arxiv15}.  In all these methods, attention aligns a sequence of input
features with words of an output sentence. However, in our work we use
attention to identify the most releavant person to the overall event during
different phases of the event.  Further, in our setting the attended set of
player detections changes between frames. This leads to interesting model choices.

\noindent \textbf{Action recognition datasets}
Action recognition in videos has eveolved with the introduction of more
sophisticated datasets starting from smaller KTH \cite{KTH}, HMDB \cite{HMDB}
to larger , UCF101 \cite{UCF101}, TRECVID-MED \cite{MED11} and Sports-1M \cite{Karpathy_CVPR14}
datasets.
More recently, THUMOS \cite{THUMOS} and ActivityNet \cite{ActivityNet} also provide a detection
setting with temporal annotations for actions in untrimmed videos.
There are also fine-grained datasets
in specific domains such as MPII cooking \cite{Finegrained_cooking} and breakfast \cite{Breakfast}.
However, most of these dataset focus on single-person activities with hardly
any need for recognizing the people responsible for the event. On the other
hand, publicly available multi-person activity datasets \cite{Choi_ICCV09,Ryoo_10} are restricted
to a very small number of videos.  One of the contributions of our work is 
a multi-player basketball dataset with dense temporal event annotations in
long videos. We provide more details in the next section.
